\documentclass{article}
\usepackage[utf8]{inputenc}
\def\MakeUppercaseUnsupportedInPdfStrings{\scshape}
\usepackage{hyperref}

\title{Aubo Guides and Training material}
\author{Lead Robotics}
\date{May 2022 v0.1.0}

\begin{document}

\maketitle

\tableofcontents


\section{This document} 
This document is a manual for setting up the Aubo for the WheelRestore System. 
The steps for a correct installation are. 
\begin{itemize}
\item Change keyboard layout (optional)
\item Set static ip to 192.168.2.41, mask: 255.255.255.0
\item Update Aubo robot to version 4.5.49, in order to enable TCP control. 
\item Install the HBCPaint program on the robot, using a backup file
\item Setup remote access with VNC (optional)
\end{itemize}

You will need a USB with the update software. The USB should also have a backup of the demo-machine, you will use to install the robotprograms, setup modbus etc.

\section{Guide}
\subsection{Startup}
\label{subsec:Startup}
Another description of the startup procedure can be found in the Aubo manual in the section 'Getting Started'.

TLDR: Plug everything in. Turn the switch on the controlbox. Wait until the standby light starts. Then hold the power button on the teachpendant for around 2 seconds. 

After startup the Aubo programming environment (AuboPE) will start automatically. Login with the default password: '1'. Press save, and then startup on the init form page. The manipulator arm should now power on, the brakes will click, and the AuboPE UI should appear.  

step by step:
\begin{itemize}
  \item Connect the three cables to the controlbox. Connect the other ends to the manipulater arm, 230VAC power and the teacpendant screen respectively.
  \item Turn the box on. On the compact box the switch is located right above the power cord plugin. On the large box the breaker is round and jutting out of the front of the box. After turning the switch you should hear the fans start in the box.
\item Wait for the ‘Standby’ light to turn on with a solid orange light. 
\item If the Emergency stop light is lit up in red make sure to clear the emergency stop buttons. There is one on the teachpendant and one on the large control box. 
\item Now get the teachpendant and press and hold the power button in the top-left corner for at least 1s. The button should light up blue.
\item Wait for the teachpendant to start. 
\item A guarantee disclaimer might appear, click agree to continue. 
\item You should now see a login screen and a desktop. 
\item Close the login screen for now, we will get to the robot program later. 
\item You should now only see the desktop with 3 shortcuts -> ‘AUBOPE’, ‘Files’ and ‘Terminal’. AUBOPE is the robot software. This will re-open the login screen and get us into the robot programming. Files opens the filesystem. Terminal opens a bash terminal.
\end{itemize}

\subsection{Shutdown} 
Press and hold the power button on the topleft of the teachpendant for at least a second.

The Screen should go black. 

The Robot is ready to start back up when the ‘standby’ light on the control box lights up orange. 

\subsection{Logging in}
\label{subsec:login}
After restart the login screen should appear automatically. Otherwise doubleclick on the AUBOPE shortcut on the desktop. Type the password, default is '1' and press login. 

The next screen is called 'Robot Init Form', press save and then startup to proceed. The robot will now go through an initializing. You should hear the brakes click.

The teachpendant should then show the robot software user interface.  

\subsection{opening system settings}
\label{subsec:OpenSystemSettings}
The robot runs on ubuntu 14.04, and there are a number of reasons you might want to access the ubuntu settings. Fx setting up IP's and changing keyboard layout. 

\begin{itemize}
	\item Startup the robot and close the login screen so you are looking at the desktop
	\item Open a terminal, by double tapping the icon. (or plug in a mouse and doubleclick on it)
	\item Enter command: $$'unity\&’$$, beware if you haven't changed the keyboard layout the $$‘\&’$$ sign will be located on $$'shift+7’$$ as on a US keyboard not on $$‘shift+6’$$ as with a Danish keyboard. This command will start the unity desktop environment and give you a toolbar on the left side of the screen.
	\item Open system settings either on the toolbar to the left or with the dropdown menu in the topright corner of the screen. The icon is a gear and a wrench. 
\end{itemize}

\subsection{Changing Keyboard Layout}
The keyboard layout will normally be set to English(US). This might be fine but be aware, if you are using a different keyboard, that some keys might be placed differently. If you want to change the layout continue in this section.

\begin{itemize}
\item startup and open system settings. See sections \ref{subsec:Startup} and \ref{subsec:OpenSystemSettings}
\item In system settings open ‘Text Entry’
\item Under the list of input sources on the left press the ‘+’ button to add an input source. 
\item Find your desired layout in the list (fx. Danish). 
\item Select your layout and press add. 
\item The name of your layout should now appear in the list of input sources marked with orange.
\item Close the Text Entry box.
\item Go to the top right corner of the screen and click where it says ‘En’ for English. This will open a dropdown menu where you can now select your newly added keyboard layout. 
\item You should now have a different keyboard configuration. Test it by opening a terminal and typing some symbols.
\end{itemize}

\subsection{Set static IP}
There are two main ways to change the static IP. 

You can write changes directly in the file /etc/network/interfaces or you can do it through system settings. 

It is recommended to do it through system settings. 

\subsubsection{Using System Settings}
\begin{itemize}
\item Start robot and open system settings (see \ref{subsec:Startup} and \ref{subsec:OpenSystemSettings})
\item Doubleclick on network
\item Select wired. The icon with the ethernet port not the arrows.
\item click options
\item Select the IPv4 settings tab
\item In the Method dropdown bar select $'manual'$ 
\item Click add and type in your desired IP, mask and gateway. 
\item Click save. 
\end{itemize}
The robot should now have a static IP. You can test it by connecting a computer and pinging the robot on its new adress. 

\subsubsection{Using interfaces file}
Open a terminal and type the command $ifconfig$. 
This will list your connection. Find the name of your ethernet connection fx. $'eth3'$ and write out the ethernet specifications. An example could be. 
\#eth3
auto eth3 
iface eth3 inet static 
address 192.168.137.2 \#static Ip of robot
gateway 192.168.137.1 
netmask 255.255.255.0  
\#eth3 config finished
navigate to /etc/network/interfaces, and open it with a text editor. 
add your specification at the bottom of the file.
Now either restart the network config, or the entire system.

\subsection{update}
There are two ways of doing this. Automatically through AuboPE or manually.
It is recomended to do it automatically. 
Check out the guides \href{https://drive.google.com/drive/folders/1e2sAyCd5S1s4jH7FRyMwZzTy7VTZb2NE}{here}. For more info. 

\subsubsection{Update via USB}
\begin{itemize}
\item Acquire USB with update software. the software is a compressed file ending in .aubo. 
\item plugin the usb. 
\item Startup and login to the aubo robot. See \ref{subsec:Startup} and \ref{subsec:login} 
\item Click on the settings tab at the top. 
\item select $'system'$ in the bottom left. 
\item Select update on the left. 
\item Click on the scan Software Package button in the center of the screen. the names of the update files on your USB should appear in the list.
\item Select your desired update file and press $'Update Software'$
\item A confirmation window will appear. Press $'yes'$ you do want to update.
\item Wait for the update to finish. This will take about 30 seconds. 
\item The program will tell you to restart. Click $'ok'$, and then $'yes'$ to shutdown.
\item Start the robot back up
\item The robot might have switched language. Dont panic. The program is still the same the text is just different. 
\item login. 
\item if the language is different go to settings and then to system. Use the previous screenshots in the guide to locate the buttons. The screen shows a sphere made of flags. Use the dropdown menu to change languages.
\item Go to the about tab and in the top right
\item Check that the Teachpendant Version has been updated. 
\end{itemize}

\subsection{Using Backups} 
With backups you can transfer all programs, variables and settings onto a new system. This 

\subsubsection{Creating a Backup}
\begin{itemize}
\item plugin a usb. 
\item Startup and login to the aubo robot. See \ref{subsec:Startup} and \ref{subsec:login}
\item Go to the settings tab.
\item open the system tab on the left
\item Select update on the left
\item select the file export tab. 
\item click $'Scan Device'$. Your usb should appear n the list. 
\item Select your USB and then press $'File Export'$
\end{itemize}

\subsubsection{Using a Backup}
\begin{itemize}
\item Acquire backupfile. Either from earlier or from a different machine you want to duplicate.
\item Move file to USB and plug USB into the robot.
\item Startup and login to the aubo robot. See \ref{subsec:Startup} and \ref{subsec:login}
\item Go to the settings tab.
\item open the system tab on the left
\item Select update on the left
\item Select the $'File Import'$ tab.
\item Click $Scan Aubo File$ and a list of the backup files on your USB should appear. 
\item Select the desired backup and press $'File Import'$
\item wait a moment for the backup to install. 
\item restart system. 
\item Check that your programs, variables, etc have been properly created.
\end{itemize}

\end{document}